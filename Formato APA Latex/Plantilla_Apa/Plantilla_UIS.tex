% ------------------------------------------------------------------------
%
% -------------------      Plantilla_UIS.tex       -----------------------
%
% ------------------------------------------------------------------------
% ------------------------------------------------------------------------
% ------------------------------------------------------------------------
% Versi�n de plantilla para realizaci�n de informes de trabajo de grado
% construida para uso de la Universidad Industrial de Santander.
%
% Reservados todos los derechos
%
% Bucaramanga, Colombia
%
% Febrero 03 de 2019
%
% ------------------------------------------------------------------------
% ------------------------------------------------------------------------
% ------------------------------------------------------------------------
%
% ------------------------------------------------------------------------
\documentclass[letter,oneside,12pt,spanish]{report}          % Encabezados
% ------------------------------------------------------------------------
\usepackage{uislatexstyleAPA}                           % Libreria UIS APA
% ------------------------------------------------------------------------
% Ingrese en este punto las librer�as espec�ficas de usuario
% ------------------------------------------------------------------------
\usepackage{epsfig}
\usepackage{amsmath}
\usepackage{amssymb}
\usepackage{subfigure}
\usepackage{hyperref}
% ------------------------------------------------------------------------
\begin{document}                                     % Inicio de documento
% ------------------------------------------------------------------------
% Definici�n sil�bica de palabras
% ------------------------------------------------------------------------
\hyphenation{pro-por-cio-nal di-se-�o}
\hypersetup {pdfborder = {0 0 0}}
% ------------------------------------------------------------------------
% Titulo resumido del trabajo que aparecer� en cornisa
% ------------------------------------------------------------------------
\title{CONJUNTOS ESTABILIZANTES EN CONTROLADORES PI}
% ------------------------------------------------------------------------
% Elementos previos al contenido del trabajo
% ------------------------------------------------------------------------
% ------------------------------------------------------------------------
%                               Portadilla
% ------------------------------------------------------------------------

\begin{center}

Verificaci�n de Conjuntos Estabilizantes para el M�todo de Dise�o de Controladores PI de Ziegler \& Nichols \vspace{2.3cm}

Emerson Rey Ardila\\ \vspace{2.3cm}

Trabajo de Grado para optar al t�tulo de Ingeniero Electr�nico\\ \vspace{2.3cm}

Director\\
Ricardo Alzate Casta�o\\
Doctorado en Ingenier�a Inform�tica y Autom�tica \vspace{2.4cm}

Universidad Industrial de Santander\\
Facultad de Ingenier�as Fisicomec�nicas\\
Escuela de Ingenier�as El�ctrica, Electr�nica y de Telecomunicaciones\\
Bucaramanga\\
2019\\

\end{center}

             % Portadilla
% ------------------------------------------------------------------------
\centering \textbf{Dedicatoria}

%% Aquí va la dedicatoria





\newpage                                   % Dedicatoria
% ------------------------------------------------------------------------
\centering \textbf{Agradecimientos}

%% Aquí van los agradecimientos





\newpage                               % Agradecimientos
% ------------------------------------------------------------------------
\tableofcontents                                      % Tabla de contenido
% ------------------------------------------------------------------------
\listoffigures                         % Lista de figuras, tablas y anexos
\listoftables
\listofanexo
% ------------------------------------------------------------------------
% ------------------------------------------------------------------------
% ------------------------------------------------------------------------
% ------------------------------------------------------------------------
%                                Glosario
% ------------------------------------------------------------------------
% ------------------------------------------------------------------------
% ------------------------------------------------------------------------
\chapter*{Glosario}

\begin{description}
  \item[Controlador] (o tambi�n compensador) es un dispositivo que toma una decisi�n con base en la comparaci�n de la informaci�n
  medida con respecto a condiciones deseadas de operaci�n. A dicha decisi�n se le denomina acci�n de control.
  \item[Controlar] es asignar valores a la variable manipulada para lograr que la variable controlada siga un valor de referencia.
  \item[Perturbaci�n] se�al indeseada que afecta negativamente el valor de la variable controlada del sistema.
  \item[PID] sigla que refiere la acci�n combinada de control proporcional, integral y derivativo.
  \item[Sistema] conjunto de elementos que interact�an de manera organizada para cumplir con un fin u objetivo com�n.
  \item[Variable Controlada] es la cantidad o condici�n que se mide y controla.
  \item[Variable Manipulada] es la cantidad que el controlador modifica para afectar los valores de salida de la planta.
\end{description}
% ------------------------------------------------------------------------                              % Glosario de t�rminos
% ------------------------------------------------------------------------
% Contenido del Informe
% ------------------------------------------------------------------------
% ------------------------------------------------------------------------
% ------------------------------------------------------------------------
% ------------------------------------------------------------------------
%                                Resumen
% ------------------------------------------------------------------------
% ------------------------------------------------------------------------
% ------------------------------------------------------------------------
\chapter*{Resumen}

\footnotesize{
\begin{description}
  \item[T�tulo:] Verificaci�n de Conjuntos Estabilizantes para Dise�o de Controladores PI de Ziegler \& Nichols \astfootnote{Trabajo de grado}
  \item[Autor:] Emerson Rey Ardila \asttfootnote{Facultad de Ingenier�as F�sico-Mec�nicas. Escuela de Ingenier�as El�ctrica, Electr�nica y telecomunicaciones. Director: Ricardo Alzate Casta�o, Doctorado en Ingenier�a Inform�tica y Autom�tica.}
  \item[Palabras Clave:] Conjunto Estabilizante, Controladores PI, Dise�o Gr�fico de Compensadores, M�todo de Ziegler \& Nichols.
  \item[Descripci�n:] El presente proyecto presenta el c�lculo de conjuntos estabilizantes para sistemas SISO LTI controlados por compensadores de estructura simple. Se estudia la fragilidad de controladores PI calculados empleando el m�todo \emph{Ziegler \& Nichols}, t�cnica de referencia en m�ltiples aplicaciones de ambito industrial. A partir de la definici�n para una m�trica basada en la interpretaci�n geom�trica para los m�rgenes de estabilidad del sistema controlado, se verifica que el controlador dise�ado con el m�todo en cuesti�n no necesariamente tolera variaciones significativas en sus valores de par�metro. Por el contrario, asume comportamientos cercanos a los l�mites de estabilidad proporcionados mediante el c�lculo de su conjunto estabilizante. Lo anterior se convierte en informaci�n importante tomando en cuenta que generalmente los m�todos de dise�o se someten a un ajuste fino. Como m�trica, se define el espacio planar correspondiente con la intersecci�n entre una elipse y una l�nea recta que representan lugares geom�tricos de m�rgen de fase y/o ganancia constantes. Adicional a lo anterior, se desarroll� una interfaz en MATLAB que permite calcular gr�ficamente los par�metros del controlador a partir de un conjunto admisible de especificaciones con base en su conjunto estabilizante. Trabajo complementario incluye la utilizaci�n de t�cnicas computacionales para el c�lculo de
conjuntos estabilizantes sobre plantas arbitrarias.
\end{description}}\normalsize
% ------------------------------------------------------------------------                                                % Resumen
% ------------------------------------------------------------------------
% ------------------------------------------------------------------------
% ------------------------------------------------------------------------
%                                Abstract
% ------------------------------------------------------------------------
% ------------------------------------------------------------------------
% ------------------------------------------------------------------------
\chapter*{Abstract}

\footnotesize{
\begin{description}
  \item[Title:] Stabilizing Set Verification for the Ziegler \& Nichols PI Controller's Design Method\astfootnote{Bachelor Thesis}
  \item[Author:] Emerson Rey Ardila\asttfootnote{Facultad de Ingenier�as F�sico-Mec�nicas. Escuela de Ingenier�as El�ctrica, Electr�nica y telecomunicaciones. Director: Ricardo Alzate Casta�o, Doctorado en Ingenier�a Inform�tica y Autom�tica.}
  \item[Keywords:] Graphical Design of Compensators, PI Controllers, Stabilizing Set, Ziegler \& Nichols Classical Method.
  \item[Description:] In this work the calculation for stabilizing sets of SISO LTI plants controlled by single structure compensators is presented. In particular, the fragility of PI controllers calculated by employing the classical method of \emph{Ziegler \& Nichols} is tested by defining a measure based on geometrical interpretation of margins of stability for the controlled loop. As we already now, the \emph{Ziegler \& Nichols} method is a reference technique for many industrial applications in practice but surprisingly after tuning parameters for desired performance the controlled system sometimes will operate riskily closed to stability boundaries determined after calculation of its corresponding stabilizing set. As a measure of the distance to instability, the intersection sets between an ellipse and a straight line constructed for constant gain and/or phase margin, is proposed. This measure shows that the controller calculated is not tolerant to changes in the parameter values, a situation typical in practice by the so-called fine tuning procedures. Also, a software interface was developed to perform graphical calculation of controller parameters using the achievable specifications set obtained from the stabilizing set of the controlled system. Ongoing work includes the automatic calculation for stabilizing sets in arbitrary plants by using computational tools already developed for that goal.
\end{description}}\normalsize
% ------------------------------------------------------------------------                                               % Abstract
% ------------------------------------------------------------------------
% Cap�tulos
% ------------------------------------------------------------------------
\input{Secs/T0}   % Introducci�n
\include{Secs/T1}   % Objetivos
\include{Secs/T2}   % Conjunto estabilizante para sistemas LTI
\input{Secs/T3}   % Modelo matem�tico del DRON
\input{Secs/T4}   % Controladores PI y su conjunto estabilizante
\input{Secs/T5}   % Recomendaciones
\input{Secs/T6}   % Trabajo futuro
\input{Secs/T7}   % Conclusiones
% ------------------------------------------------------------------------
% Bibliograf�a
% ------------------------------------------------------------------------
\addcontentsline{toc}{chapter}{Referencias Bibliogr�ficas}\newpage
\bibliographystyle{apalike}
\bibliography{xbib}
% ------------------------------------------------------------------------
% Anexos
% ------------------------------------------------------------------------
\input{Secs/anexoA} % Fundamentos de s�lidos r�gidos
\input{Secs/anexoB} % Funci�n ode45 de MATLAB
\input{Secs/anexoC} % Interfaz de animaci�n de la din�mica del sistema
% ------------------------------------------------------------------------
\end{document}                                          % Fin de documento
% ------------------------------------------------------------------------ 