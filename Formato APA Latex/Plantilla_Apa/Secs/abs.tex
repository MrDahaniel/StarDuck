% ------------------------------------------------------------------------
% ------------------------------------------------------------------------
% ------------------------------------------------------------------------
%                                Abstract
% ------------------------------------------------------------------------
% ------------------------------------------------------------------------
% ------------------------------------------------------------------------
\chapter*{Abstract}

\footnotesize{
\begin{description}
  \item[Title:] Stabilizing Set Verification for the Ziegler \& Nichols PI Controller's Design Method\astfootnote{Bachelor Thesis}
  \item[Author:] Emerson Rey Ardila\asttfootnote{Facultad de Ingenier�as F�sico-Mec�nicas. Escuela de Ingenier�as El�ctrica, Electr�nica y telecomunicaciones. Director: Ricardo Alzate Casta�o, Doctorado en Ingenier�a Inform�tica y Autom�tica.}
  \item[Keywords:] Graphical Design of Compensators, PI Controllers, Stabilizing Set, Ziegler \& Nichols Classical Method.
  \item[Description:] In this work the calculation for stabilizing sets of SISO LTI plants controlled by single structure compensators is presented. In particular, the fragility of PI controllers calculated by employing the classical method of \emph{Ziegler \& Nichols} is tested by defining a measure based on geometrical interpretation of margins of stability for the controlled loop. As we already now, the \emph{Ziegler \& Nichols} method is a reference technique for many industrial applications in practice but surprisingly after tuning parameters for desired performance the controlled system sometimes will operate riskily closed to stability boundaries determined after calculation of its corresponding stabilizing set. As a measure of the distance to instability, the intersection sets between an ellipse and a straight line constructed for constant gain and/or phase margin, is proposed. This measure shows that the controller calculated is not tolerant to changes in the parameter values, a situation typical in practice by the so-called fine tuning procedures. Also, a software interface was developed to perform graphical calculation of controller parameters using the achievable specifications set obtained from the stabilizing set of the controlled system. Ongoing work includes the automatic calculation for stabilizing sets in arbitrary plants by using computational tools already developed for that goal.
\end{description}}\normalsize
% ------------------------------------------------------------------------ 