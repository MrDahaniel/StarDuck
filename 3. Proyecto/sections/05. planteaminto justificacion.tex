\subsection{Planteamiento del problema}

La complejidad de los sistemas computacionales tiene origen en diversos factores. El aumento de la cantidad de dispositivos que los componen; la heterogeneidad debido a diferentes marcas, protocolos y características; e incluso las cambiantes condiciones de sus entornos de ejecución; dificulta la administración de los sistemas computacionales \cite{emerging_2005}.

Una de las posibles maneras de dar solución a esta problemática, en cuanto al manejo de sistemas, está en el área de la computación autonómica. Este acercamiento basado en conceptos biológicos busca solventar los problemas de complejidad, heterogeneidad e incertidumbre \cite{emerging_2005} a partir de la abstracción de las metas de los administradores y delegación del manejo del sistema a sí mismo \cite{lalanda_diaconescu_mccann_2014}.

Considerando lo anterior, una de las aplicaciones de la computación autonómica, dentro del IoT, son los Smart Campus; una variación de las Smart Cities en las cuales se busca la recolección de información y monitoreo en tiempo real con el fin de apoyar la toma de decisiones, mejora de servicios, entre otros \cite{MinAllah2020}. Es en este tipo de aplicaciones donde los problemas, en especial la heterogeneidad, dinamicidad, al igual que la distribución geográfica; donde la reducción de la dependencia de intervención humana, facilitaría el manejo de estos. 

\subsection{Justificación del problema} 

De lo anterior, surge la pregunta del cómo realizar dichas adaptaciones a la arquitectura, o estado del sistema. Qué clase de mecanismos y requerimientos han de ser satisfechos con el fin de considerar, a un sistema computacional, con la capacidad de auto-adaptarse dados unos objetivos o metas establecidas por los administradores del sistema.

Dentro del marco del presente proyecto, se tiene Smart Campus UIS, una plataforma de IoT de la Universidad Industrial de Santander, en la cual se han realizado implementaciones parciales de una arquitectura autonómica con capacidad de auto-describirse \cite{msc_henry_2022}, una de las características principales de un sistema autonómico \cite{horn_2001}. 

Dicho esto, se busca explorar algunas de las maneras en las que se realizan adaptaciones con características autonómicas en sistemas computacionales. De esto, se buscaría probar un conjunto de estos mecanismos de modificación del estado, tomando como caso de estudio la implementación de estos en la plataforma de Smart Campus UIS.