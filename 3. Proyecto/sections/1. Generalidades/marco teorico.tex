\section{Marco Teórico}

% \subsection{Notación, Sintaxis y Lenguajes}

\subsection{Notación}

De manera general, notación se refiere a una serie de caracteres, símbolos, figuras o expresiones usadas con el fin de expresar un sistema, método o proceso \cite{MerriamWebsterNotation}. Más específicamente en el contexto de las ciencias de la computación, las notaciones han sido usadas para la representación de estructuras y arquitecturas en el software, como lo son lenguajes de modelado tales como UML \cite{WhatIsUML}; lenguajes de programación, como C/C++, Ruby \cite{Bansal2013}; algoritmia de manera visual \cite{RutanenKalle2018McoO}; entre otros.


% El concepto de \textit{notación} está definido como la representación gráfica del habla. En el contexto de las ciencias de la computación, esta idea se ha extrapolado con el fin de representar diferentes conceptos específicos del software y algoritmia de manera visual \cite{RutanenKalle2018McoO}. Esto puede verse con la existencia de lenguajes de notación como lo es UML con el cual se realizan representaciones que van desde arquitecturas de software, estructuras de base de datos, entre otros \cite{Booch2005-xu}.

\subsubsection{Lenguajes}

En el contexto de la computación

\subsubsection{Gramática}

% Esto es más que nada para contextualizar la necesidad de una forma en la notación

La gramática, más específicamente gramáticas libres de contexto, son un conjunto de reglas descriptivas. Este conjunto de reglas, en conjunto de una notación, cumplen la función de dictar si una frase es válida para un lenguaje dado \cite[p. 101]{Sipser2012-wl}. 

\subsubsection{Sintaxis}


\subsubsection{Domain-Specific Languages} % Fase 2

% En esta selección es más que todo el dar el concepto de qué es un lenguaje de notación

% Explicar que es la notación en general, en términos lingüísticos y de ahí expandir a notación en software
% UML y eso.

Los \textit{Domain-Specific Languages} (DSL), o Lenguaje de dominio específico, son lenguajes de programación los cuales se usan con un fin específico. Estos están orientados principalmente al uso de abstracciones de un mayor nivel debido al enfoque que estos tienen para la solución de un problema en específico \cite{Kelly2008}. Ejemplos de este tipo de acercamiento, en el caso del modelado (\textit{Domain-specific modeling}), pueden observarse en los diagramas de entidad relación usados en el desarrollo de modelos de base de datos \cite{Celikovic2014ADF}. 

\subsection{Serialización de Datos}

La serialización de datos se refiere a la traducción de una estructura de datos a una manera en la que pueda ser almacenada o transmitida de manera eficiente \cite{mozillaSerialization}. Este proceso ha sido, principalmente, adoptado por lenguajes de programación orientada a objetos en donde existe un soporte nativo, tales como Go, JavaScript, o PHP; o existe algún tipo de framework o librería que permite hacerlo, como lo es en el caso de C/C++, Rust o Perl.

\subsubsection{Métodos de Serialización de Datos}

Dentro del ámbito de la programación, se encuentran diversas opciones en cuanto a formatos y técnicas empleadas para la serialización de datos. Cada uno de estos enfoques presenta sus particularidades, lo que resulta en ventajas y desventajas específicas las cuales deben ser consideradas según las necesidades del proyecto  