\section{Estado del Arte}

Con el objetivo de explorar a fondo el panorama actual de la computación autonómica, y en particular, los mecanismos de adaptación de arquitecturas de software y los requisitos fundamentales para su implementación, se llevó a cabo una revisión de la literatura en diversas bases de datos. Esta revisión abarcó un recorrido que partió de una visión general y se adentró en aspectos cada vez más específicos. Durante este proceso, se examinaron detalladamente las propuestas y componentes clave de sistemas de software autonómicos, así como las diversas nociones relacionadas con notaciones, algoritmos para la comparación de estructuras de datos y los mecanismos esenciales para la adaptación de arquitecturas de software. 

\subsection{Propuestas de Sistemas de Software Autonómicos}




\subsection{Notación, Sintaxis y Lenguajes de Dominio Específico}


\subsection{Algoritmos para la Comparación de Estructuras de Datos}


\subsection{Mecanismos de Adaptación en Arquitecturas de Software}