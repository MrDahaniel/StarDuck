\subsection{La necesidad de una arquitectura de referencia}

La necesidad de una arquitectura de referencia, o arquitectura objetivo; es una parte esencial dentro de la computación autonómica. Esta forma parte de la base de conocimiento (K), y, de manera indirecta, de los objetivos del administrador del sistema \cite[p. 24]{lalanda_diaconescu_mccann_2014}. 

Con el fin de fijar un objetivo para el sistema autonómico, fue necesario determinar una manera de realizar la declaración de dicha arquitectura, que estableciera un estado de referencia. De esta manera, podría evaluarse el estado del sistema en tiempo de ejecución,  y así tomar las acciones necesarias para adaptarlo hacia el estado de referencia. 

% \subsection{La noción de aplicación}

Ahora, es necesario establecer el punto desde el cual se realizará la comparación entre los estados del sistema. Esto guiará la búsqueda para determinar el como se realizará la declaración de los estados objetivo de Smart Campus UIS, al igual que los datos que se recolectarán.

Siendo así, y dados los objetivos que buscan cumplir los ecosistemas inteligentes para la toma de decisiones \cite{Anagnostopoulos_2023}, se estableció que la métrica por la cual se describiría, y evaluaría, el estado del sistema sería a partir de los datos que estaban siendo recolectados. Es decir, que el enfoque debía estar orientado a cumplir con las necesidades de los datos establecidas por Smart Campus. 