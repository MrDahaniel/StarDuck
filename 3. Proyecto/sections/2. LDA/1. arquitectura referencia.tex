\subsection{La necesidad de una arquitectura de referencia}

La necesidad de una arquitectura de referencia, o arquitectura objetivo; es una parte esencial dentro de la computación autonómica. Esta forma parte de la base de conocimiento (K), y, de manera indirecta, de los objetivos del administrador del sistema \cite[p. 24]{lalanda_diaconescu_mccann_2014}. 

Con el fin de establecer un objetivo para el sistema autonómico, fue necesario determinar una manera de realizar la declaración de dicha arquitectura, que estableciera un estado de referencia. De esta manera, podría evaluarse el estado del sistema en tiempo de ejecución,  y así tomar las acciones necesarias para adaptarlo hacia el estado de referencia. 

Ahora, fue también necesario establecer el punto desde el cual se lleva a cabo la comparación entre los estados del sistema. Esto guió la búsqueda para determinar el como se realizó la declaración de los estados objetivo de Smart Campus UIS, al igual que los datos a recolectar.

Siendo así, y dados los objetivos que buscan cumplir los ecosistemas inteligentes para la toma de decisiones \cite{Anagnostopoulos_2023}, se enfocó no sobre la arquitectura sobre la cual trabaja la plataforma, sino sobre las aplicaciones desarrollándose sobre esta. Es decir, los datos requeridos. Son las necesidades definidas por las aplicaciones, las descritas y usadas para evaluar el estado del sistema. Siendo así, que el enfoque debía estar orientado a cumplir con las necesidades de los datos establecidas por Smart Campus. Esto se expondrá de manera más clara durante las fases de comparación y adaptación.