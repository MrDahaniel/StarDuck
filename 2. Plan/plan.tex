\documentclass[12pt]{article}

% Importando Config
\usepackage[spanish]{babel}
\usepackage{apacite}

\addto\captionsspanish{%
  \renewcommand{\tablename}{Tabla}
}

\usepackage[table,xcdraw]{xcolor}

\usepackage[T1]{fontenc}
% \usepackage{uarial}
% \renewcommand{\familydefault}{\sfdefault}

\usepackage{etoolbox}
\patchcmd{\thebibliography}{\section*{\refname}}{}{}{}


\usepackage{url}
\def\UrlBreaks{\do\/\do-}

\usepackage{adjustbox}
\usepackage{graphicx}
\usepackage{tabularx}
\usepackage{svg}
\usepackage{float}
\restylefloat{table}

\usepackage{enumitem}

\usepackage{multicol}

\parskip=12pt 
\parindent=0pt

\usepackage{ragged2e}
\tolerance=1
\emergencystretch=\maxdimen
\hyphenpenalty=10000
\hbadness=10000
\raggedright

\usepackage{textcase}
\usepackage{tocloft}
\makeatletter
\patchcmd{\l@section}{#1}{\MakeTextUppercase{#1}}{}{}
\patchcmd{\l@subsection}{#1}{\MakeTextUppercase{#1}}{}{}
\patchcmd{\l@subsubsection}{#1}{\MakeTextUppercase{#1}}{}{}
\makeatother

\usepackage{titlesec}
\titleformat{\section}{\raggedright\normalfont\normalsize\bfseries\uppercase}{\thesection}{1em}{}
\titleformat{\subsection}{\raggedright\normalfont\normalsize\bfseries\uppercase}{\thesubsection}{1em}{}
\titleformat{\subsubsection}{\raggedright\normalfont\normalsize\bfseries\uppercase}{\thesubsubsection}{1em}{}

\usepackage{geometry}
\geometry {
    letterpaper,
    left = 1in,
    right = 1in,
    bottom = 1in,
    top = 1in
}

% Line break
\newcommand{\skipline}{\par\null\par}

\usepackage{subfig}

\newcommand*{\Universidad}[1]{\def\Uni{#1}}
\newcommand*{\Facultad}[1]{\def\Fac{#1}}
\newcommand*{\Escuela}[1]{\def\Esc{#1}}

\usepackage{datetime}
\newdateformat{daymonthyear}{\THEDAY \ de \monthname[\THEMONTH] de \THEYEAR}
\newcommand*{\CiudadFecha}{Bucaramanga, \daymonthyear\today }

\newcommand*{\Titulo}[1]{\def\Tit{#1}}
\newcommand*{\Modalidad}[1]{\def\Mod{#1}}
\newcommand*{\Autor}[2]{\def\Nam{#1}\def\Cod{#2}}
\newcommand*{\Director}[3][]{\def\TDir{#1}\def\Dir{#2}\def\EDir{#3}}
\newcommand*{\CoDirector}[3][]{\def\TCDir{#1}\def\CDir{#2}\def\ECDir{#3}}
\newcommand*{\EntidadInt}[1]{\def\EntI{#1}}

\newcommand*{\captionsource}[2]{%
  \caption[{#1}]{%
    #1%
    \\\hspace{\linewidth}%
    \textbf{Source:} #2%
  }%
}

\titlespacing\section{0pt}{12pt plus 4pt minus 2pt}{0pt plus 2pt minus 2pt}
\titlespacing\subsection{0pt}{12pt plus 4pt minus 2pt}{0pt plus 2pt minus 2pt}
\titlespacing\subsubsection{0pt}{12pt plus 4pt minus 2pt}{0pt plus 2pt minus 2pt}


\bibliographystyle{apacite}

% Definiendo constantes
\Universidad{Universidad Industrial de Santander}
\Facultad{Facultad De Ingenierías Fisicomecánicas}
\Escuela{Escuela De Ingeniería De Sistemas E Informática}
\Titulo{Mecanismos de adaptación autonómica de arquitectura software para la plataforma Smart Campus UIS}
\Modalidad{Trabajo de investigación}
\Autor{Daniel David Delgado Cervantes}{2182066}
\Director[PhD.]{Gabriel Rodrigo Pedraza Ferreira}{Escuela De Ingeniería De Sistemas e Informática}
\CoDirector[MSc.]{Henry Andres Jimenez Herrera}{Escuela De Ingeniería De Sistemas e Informática} 
\EntidadInt{Universidad Industrial de Santander}

% En el caso de no tener codirector, quitar la linea `\textbf{CODIRECTOR: } \TCDir\ \CDir, \ECDir` de 'Sections/title.tex'

\begin{document}

    % maketitle page
    % --------------------------------------------------------------------------------------------------------- %
%                                               Section: Title                                              %
% --------------------------------------------------------------------------------------------------------- %

\renewcommand{\contentsname}{\hfill\bfseries\normalsize \MakeUppercase{Tabla de Contenido}\hfill}
\renewcommand{\cftaftertoctitle}{\hfill}

\begin{titlepage}
        
    \begin{center}

        \textbf{\MakeUppercase{\Uni}} \\
        \textbf{\MakeUppercase{\Fac}} \\
        \textbf{\MakeUppercase{\Esc}}
        
        \skipline
        
        \textbf{PLAN DE TRABAJO DE GRADO}
        
        \skipline
        \skipline
        
    \end{center}
    
    
    \textbf{FECHA DE PRESENTACIÓN: } \CiudadFecha

    \textbf{TÍTULO: } \Tit
    
    \textbf{MODALIDAD: } \Mod

    \textbf{AUTOR: } \Nam, \Cod

    \textbf{DIRECTOR: } \TDir\ \Dir, \EDir
    
    \textbf{CODIRECTOR: } \TCDir\ \CDir, \ECDir

    \textbf{ENTIDAD INTERESADA: } \EntI 
    
\end{titlepage}

\tableofcontents

\pagebreak
\begin{center}
    \MakeUppercase{\textbf{ \Tit}}
\end{center}
    
    \section{Introducción}

    % 1. Origen de la computación autonómica (Citation needed for industry tendencies)

    Dentro de la computación distribuida, una de las tendencias recientes dentro de la industria , es la búsqueda de maneras de reducir la complejidad de administrar los sistemas computacionales. A medida que estos crecen, en términos de tamaño y extensión, el costo humano de la administración de los sistemas se vuelve insustentable. En respuesta a esto, diferentes enfoques han surgido. Uno de estos es el \textit{autonomic computing} (computación autonómica). Planteada originalmente por IBM en el año 2001, se presentaba como una posible solución a la problemática a partir de el diseño y la implementación de componentes auto-gestionados \cite{jeff_2011}. 

    % 2. Relación y utilidad para las otras ramas de la computación (¿Qué ramas?, ¿Por qué les es útil?)

    Este acercamiento, aunque inicialmente concebido únicamente para sistemas distribuidos, puede también ser particularmente útil en otras ramas como lo son las 

    % 3. Argumentar su utilidad en en el caso específico de IoT (Costo humano, posibles funcionamientos degradados)
    %    Y lo que representa para Smart Campus (Dar continuidad en ese sentido)
    % 4. Presentar el objetivo final del proyecto (¿Qué se espera obtener?)
    % 5. Presentar de manera general la metodología a seguir

    \section{Planteamiento Y Justificación Del Problema}
    
    La complejidad de los sistemas software ha ido en aumento. A medida que se hace la transición a arquitecturas orientadas a microservicios \cite{forrester_research_2019}; la computación distribuida es más común gracias a las soluciones \textit{cloud} \cite{the_cloud_in_2021} y la computación embebida se hace más presente;  la administración y gestión de estos requiere de una mayor cantidad de recursos en términos técnicos y humanos con el fin de mantenerlos en los estados más óptimos respecto a los requerimientos del negocio. La búsqueda de reducir o abstraer la complejidad de la gerencia de estos sistemas se ha convertido en una necesidad \cite{lalanda_diaconescu_mccann_2014}.
    
    Esta necesidad, así mismo, se presenta en los campos del Internet de las Cosas (IoT). Es en esta área de la computación embebida donde, debido a las cambiantes condiciones del mundo real, la arquitectura de estos sistemas de software se ve constantemente afectada. Una de las posibles soluciones se encuentra en la computación autonómica. Desde este enfoque, se tiene como objetivo sistemas con la capacidad de auto-gestión, es decir, sistemas con la capacidad de manejarse a ellos mismos dependiendo de las necesidades y las metas establecidas por los administradores del sistema \cite{evaluation_2004}.
    
    Ahora, tenemos el caso de Smart Campus UIS, una plataforma de IoT de la Universidad Industrial de Santander. Esta ha realizado implementaciones parciales de una arquitectura autonómica con capacidad de auto-describirse \cite{henry_2020}. Dicho esto, y en búsqueda dar continuidad con los esfuerzos de desarrollo realizados en la plataforma, el siguiente paso a dar está en la implementación de mecanismos de adaptación los cuales le concedan las propiedades de auto-configuración y auto-sanación.

    \section{Objetivos}
    \subsection{Objetivo General}
    \begin{itemize}

        \item Implementar un conjunto de mecanismos autonómicos para permitir la adaptación de la Arquitectura Software IoT respecto a un modelo objetivo en la plataforma Smart Campus UIS

    \end{itemize}

    \subsection{Objetivos Específicos}

    \begin{itemize}
        \item Proponer una notación (lenguaje) para describir una arquitectura objetivo de un sistema software IoT.
        \item Diseñar un mecanismo para determinar las diferencias existentes entre una arquitectura actual en ejecución y una arquitectura objetivo especificada.
        \item Diseñar un conjunto de mecanismos de adaptación que permitan disminuir las diferencias entre la arquitectura actual y la arquitectura objetivo.

    \end{itemize}

    \section{Marco De Referencia}

    \subsection{Computación Autonómica}

    \subsubsection{MAPE-K}

    \subsubsection{Mecanísmos de Adaptación}

    \subsection{Computación Embebida}
    
    \subsubsection{Internet of Things}
    
    \subsection{Lenguajes de Notación} % Fase 2

    \subsubsection{Serialización de Datos}
    
    \subsubsection{Gramática}

    \subsection{Algoritmia de Comparación de Grafos} % Fase 3


    \section{Metodología}

    Para el desarrollo del trabajo de grado, se ha definido un modelo de prototipado iterativo compuesto de 5 fases (Ver fig. \ref{fig:met}). De esta manera, se avanzará a medida que se va completando la fase anterior y permitirá a futuro el poder iterar sobre lo que se ha desarrollado anteriormente.

    \begin{figure}[h]
        \centering
        \includesvg{Images/DiagramaMetodologia.svg}
        \caption{Metodología del proyecto}
        \label{fig:met}
    \end{figure}

    \subsection{Ambientación Conceptual y Tecnológica}

    La primera fase de la metodología se basa en la investigación de la literatura, al igual que de la industria, necesaria para cubrir las bases tanto conceptuales como técnicas necesarias para el desarrollo del proyecto. 

    \subsubsection*{Actividades}

    \begin{enumerate}[label=\thesubsection.\arabic*., wide, labelindent=0pt]
        \item Identificar las características principales de un sistema auto-adaptable.
        \item Analizar los mecanismos de adaptación de la arquitectura.
        \item Analizar los algoritmos empleados para la comparación de la comparación de las arquitecturas.
        \item Establecer los criterios de selección para el lenguaje de notación.
        \item Evaluar los posibles lenguajes de programación para la implementación a realizar.
        \item Imprevistos.
        \item Análisis, retroalimentación y conclusiones del desarrollo de la fase. 
    \end{enumerate} 

    % \subsubsection{Actividades}

    % \subsubsection{Productos}

    % \begin{itemize}
    %     \item Criterios para la selección del lenguaje de notación.
    % \end{itemize}

    \subsection{Definición de la notación de la arquitectura}
    
    La segunda fase está en la definición del como se realiza la declaración de la arquitectura. Partiendo de los criterios de selección establecidos el la fase 1, se espera determinar un lenguaje de notación el cual nos permita definir la arquitectura objetivo a alcanzar, al igual que la gramática correspondiente para poder realizar dicha declaración. 
    
    \subsubsection{Actividades}

    \begin{enumerate}[label=\thesubsubsection.\arabic*., wide, labelindent=0pt]
        \item Seleccionar el lenguaje de notación a usar a partir de los criterios establecidos.
        \item Definir la gramática a usar para la definición de la arquitectura según (Buscar qué estándar podemos seguir). % !!!
        \item Traducir la notación de la arquitectura a grafos (¿Esto en qué fase debería estar?)
        % \item 
    \end{enumerate}    

    \subsection{Mecanismos De Comparación}

    Durante la tercera fase del proyecto, se buscará poder determinar e implementar cómo se realizará la comparación entre el estado de la arquitectura obtenido durante la auto-descripción de la misma y el objetivo establecido. Con el fin de cumplir esto, será necesario realizar la traducción de la arquitectura desde su forma notada a un estado donde se pueda realizar el procesamiento necesario para establecer las diferencias tanto como el mecanismo responsable de dicho procesamiento.

    \subsubsection{Actividades}

    \begin{enumerate}[label=\thesubsubsection.\arabic*., wide, labelindent=0pt]
        \item Identificar un mecanismo (¿O algoritmo?) para realizar la comparación de los modelos
        \item Implementar el mecanismo seleccionado
        \item 
    \end{enumerate}    

    \subsection{Mecanismos De Adaptación}
    
    La cuarta fase del proyecto está orientada a la selección, al igual que la implementación, del conjunto de mecanismos de adaptación de la arquitectura. 

    \subsubsection{Actividades}

    \begin{itemize}
        \item Actividad 1
        \item Actividad 2
        % \item 
    \end{itemize}  

    \subsection{Validación De Resultados}


    \subsubsection{Actividades}

    \begin{itemize}
        \item Actividad 1
        \item Actividad 2
        % \item 
    \end{itemize}  

    \section{Cronograma}

    Se debe realizar un cronograma que relacione las actividades prioritarias del proyecto y el tiempo que destinará a cada una de ellas. Tenga en cuenta que el semestre tiene 16 semanas y debe desarrollar todo el trabajo de grado en este tiempo. 

    \section{Presupuesto}


    
    \begin{table}[ht]
    \small
    % \resizebox{\textwidth}{!}{%
        % \begin{tabular}{|l|c|c|c|c|}
            \begin{tabularx}{\textwidth}{|X|c|X|X|X|}
                \hline
                \multicolumn{1}{|c}{\textbf{Descripción}} & \multicolumn{1}{|c}{\textbf{Responsable}} & \multicolumn{1}{|c}{\textbf{Valor}} & \multicolumn{1}{|c}{\textbf{Cantidad}} & \multicolumn{1}{|c|}{\textbf{Precio}} \\ \hline
                DIRECTOR DE PROYECTO PhD. Gabriel Rodrigo Pedraza Ferreira & UIS & COP 305.000/Hora & \multicolumn{1}{X|}{\raggedright 4 horas mensuales por 4 meses} & COP 4'880.000 \\ \hline
        &  &  &  &  \\ \hline
        &  &  &  &  \\ \hline
        &  &  &  &  \\ \hline
        &  &  &  &  \\ \hline
        
        % \end{tabular}%
        \end{tabularx}{\parfillskip=0pt\par}
    % }
    \end{table}
    

    \pagebreak

    \section{Bibliografía}

    \begingroup
    \renewcommand{\section}[2]{}
    \renewcommand{\addcontentsline}[3]{}
    \bibliography{bibliography}
    \endgroup

\end{document} 